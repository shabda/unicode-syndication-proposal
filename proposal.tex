\documentclass[10pt,a4paper,oneside,draft]{article}
\usepackage[utf8]{inputenc}
\usepackage{amsmath}
\usepackage{amsfonts}
\usepackage{amssymb}
\usepackage{hyperref}

\author{Victor Felder \and you?}
\title{Proposal to Include Web Syndication Symbol}

\begin{document}
\maketitle

\begin{abstract}
The web syndication icon, also known as web feed icon, is not in Unicode. A good proportion\footnote{The exact proportion is very hard to estimate. If we assume most websites are using a CMS and provided that most of the CMS provide syndication feeds by default, it would be safe to assume that most websites provide a syndication feed.} of websites provide a syndication feed in a standardized format (usually RSS or Atom), see for example \href{http://unicode.org}{unicode.org}'s RSS feed. In 2005 most major browsers adopted\footnote{\url{https://en.wikipedia.org/wiki/Feed_icon\#History}} a common icon\footnote{\url{https://en.wikipedia.org/wiki/Feed_icon\#/media/File:Feed-icon.svg}} to represent web syndication feeds. This icon “is one of the 100 most-used files on the English Wikipedia.”\footnote{\url{https://en.wikipedia.org/wiki/File:Feed-icon.svg}} Electronically published documents rely either on special fonts or image files to represent this symbol. A Unicode character would be very useful in any web context to indicate the presence of syndication feeds.
\end{abstract}

\section{Introduction}
\section{Suitability for Inclusion}
\section{Evidence of Use in Running Text}
\section{Character Properties}
\section{Collation Order}
\section{Anticipated Objections}
\section{Drawing the Symbol}
\section{Sponsors}
\section{Summary and Conclusion}
\end{document}